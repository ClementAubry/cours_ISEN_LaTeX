%!TEX root = beamerCoursIsenLaTeX.tex
\documentclass[./beamerCoursIsenLaTeX.tex]{subfiles}

\begin{document}
%%%%%%%%%%%%%%%%%%%%%%%%%%%%%%%%%%%%%%%%%%%%%%%%%%%%%%%%%%%%%%%%%%%%%%%%%%%%%%
\section{Avoir de bons outils}
%%%%%%%%%%%%%%%%%%%%%%%%%%%%%%%%%%%%%%%%%%%%%%%%%%%%%%%%%%%%%%%%%%%%%%%%%%%%%%
\subsection{\LaTeX}
%%%%%%%%%%%%%%%%%%%%%%%%%%%%%%%%%%%%%%%%%%%%%%%%%%%%%%%%%%%%%%%%%%%%%%%%%%%%%%
\begin{frame}
\begin{itemize}
\item Pour bien travailler.
\item Il faut les bons outils.
\item La dernière version de \LaTeX convient tout à fait.
\end{itemize}
\begin{figure}
\begin{center}
\begin{tikzpicture}[auto, thick, node distance=2cm]
  \draw (-3,-1.5)rectangle(4,1);
  % Dessin des blocs
  \node at (-4,0.2) (cmd) {$\mathbf{w}(t)$};
  \draw[black] node at (-1,0)[block] (regul) {Régulateur};
  \draw[black] node at (2,0)[block] (system) {Système};
  \draw node at (5,0) (out) {$\mathbf{y}(t)$};
  % Joining blocks.
  \draw[->,black](cmd.east) -- ([yshift=0.2cm]regul.west);
  \draw[->,black](regul.east) -- (system.west) node at ($(regul.east)!0.5!(system.west)$)[above]{$\mathbf{u}(t)$};
  \draw[->,black](system.east) -- (out);
  \draw[->,black](3,0)--(3,-1)--(-2.5,-1)--(-2.5,-0.2)--([yshift=-0.2cm]regul.west);
\end{tikzpicture}
\caption{Exemple de dessin en Tikz.}
\end{center}
\end{figure}
\end{frame}
%%%%%%%%%%%%%%%%%%%%%%%%%%%%%%%%%%%%%%%%%%%%%%%%%%%%%%%%%%%%%%%%%%%%%%%%%%%%%%
\subsection{TexLive 2015}
%%%%%%%%%%%%%%%%%%%%%%%%%%%%%%%%%%%%%%%%%%%%%%%%%%%%%%%%%%%%%%%%%%%%%%%%%%%%%%
\begin{frame}
\begin{itemize}
\item La version 2015 de TeXLive est fortement recommandée.
\item \url{https://www.tug.org/texlive/}
\end{itemize}
  \begin{block}{Définition : Système [Larousse]}
    \begin{itemize}
      \item Appareillage, dispositif formé de divers éléments et assurant une fonction déterminée
    \end{itemize}
  \end{block}
  \null
\begin{columns}
\begin{column}{5cm}
Domaines d'application : 
\begin{itemize}
\itemsep1pt
\item électronique
\item mécanique
\item biologique
\end{itemize}
\end{column}
\begin{column}{6cm}
\begin{figure}[!ht]
\begin{center}
\begin{tikzpicture}[auto, thick, node distance=2cm, >=triangle 45]
  % Dessin blocs
  \node at (-2,0) (in) {entrées};
  \draw[black] node at (0,0)[block] (system) {Système};
  \draw node at (2,0) (out) {sorties};
  % Joining blocks.
  \draw[->,black](in) -- (system.west);
  \draw[->,black](system.east) -- (out);
\end{tikzpicture}
\end{center}
\caption{Un système}
\end{figure}
\end{column}
\end{columns}
\end{frame}
%%%%%%%%%%%%%%%%%%%%%%%%%%%%%%%%%%%%%%%%%%%%%%%%%%%%%%%%%%%%%%%%%%%%%%%%%%%%%%
\subsection{Un bon éditeur de texte}
%%%%%%%%%%%%%%%%%%%%%%%%%%%%%%%%%%%%%%%%%%%%%%%%%%%%%%%%%%%%%%%%%%%%%%%%%%%%%%
\begin{frame}
\begin{itemize}
\item Personellement j'utilise Sublime Text 3
\item \url{https://www.sublimetext.com/3}
\item Avec le plugin LatexTools
\end{itemize}
\cestimportant
\end{frame}
%%%%%%%%%%%%%%%%%%%%%%%%%%%%%%%%%%%%%%%%%%%%%%%%%%%%%%%%%%%%%%%%%%%%%%%%%%%%%%
\subsection{La compilation}
%%%%%%%%%%%%%%%%%%%%%%%%%%%%%%%%%%%%%%%%%%%%%%%%%%%%%%%%%%%%%%%%%%%%%%%%%%%%%%
\begin{frame}
\center Tout se fait à base de Makefile.\\Attention, ceux distribués sont faits sous Ubuntu\\
\begin{itemize}
\item Pour la présentation : \texttt{make all}
\item Pour les TD : \texttt{make TEX=<nomDuFichierTDSansExtension> all}
\end{itemize}

\end{frame}
%%%%%%%%%%%%%%%%%%%%%%%%%%%%%%%%%%%%%%%%%%%%%%%%%%%%%%%%%%%%%%%%%%%%%%%%%%%%%%
\end{document}

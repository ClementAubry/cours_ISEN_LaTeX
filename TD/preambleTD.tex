\usepackage[french]{babel}
\usepackage[utf8]{inputenc}
\usepackage[T1]{fontenc}
\usepackage{subfiles}
\usepackage{amssymb,amsmath,amsfonts,amsthm}%Maths
\usepackage{empheq}
\usepackage{bigints}
\usepackage{pdfpages}
\usepackage[top=2.5cm, bottom=2.5cm, left=2.5cm, right=2.5cm]{geometry}
\usepackage{fancyhdr}
\usepackage{textcomp}
\usepackage{etoolbox}
\usepackage{xcolor}
\usepackage{url}
\usepackage{placeins}
\usepackage{lastpage}
\usepackage{tikz}
\usepackage{draftwatermark}
\usetikzlibrary{intersections,shapes,arrows,calc,positioning,patterns,decorations.pathmorphing,decorations.markings,matrix,snakes}
\usepackage[european,cuteinductors]{circuitikz}
\definecolor{almond}{rgb}{0.94, 0.87, 0.8}
\definecolor{redisen}{RGB}{227,6,19}
\definecolor{grayisen}{RGB}{192,196,208}
% \definecolor{grayisen1}{RGB}{172,176,188}
\definecolor{grayisen1}{RGB}{92,96,108}
\usepackage{listings}
\lstset{frame=single,backgroundcolor=\color{grayisen!10},breaklines=true}



\lstdefinestyle{Bash}
{language=bash,
 keywordstyle=\bf\color{black},
 basicstyle=\ttfamily,
 morekeywords={username@hostname},
 alsoletter={\$:~},
 morekeywords=[2]{username@hostname:},
 keywordstyle=[2]{\color{black}},
 literate={\$}{{\textcolor{black}{\$}}}1
 {:}{{\textcolor{black}{:}}}1
 {~}{{\textcolor{black}{{\raise.17ex\hbox{$\scriptstyle\sim$}}}}}1,
 % comment=[l]{#},
 % commentstyle=\color{gray}\ttfamily,
 % stringstyle=\color{gray}\ttfamily,
}
\lstdefinestyle{cmakelist}
{language=bash,
 keywordstyle=\bf\color{black},
 basicstyle=\footnotesize \ttfamily,
 morekeywords={username@hostname},
 alsoletter={\$:~\#},
 morekeywords=[2]{username@hostname:},
 keywordstyle=[2]{\color{black}},
 literate={\$}{{\textcolor{black}{\$}}}1
 {:}{{\textcolor{black}{:}}}1
 {\#}{{\textcolor{black}{\#}}}1
 {~}{{\textcolor{black}{{\raise.17ex\hbox{$\scriptstyle\sim$}}}}}1,
}


\newcommand{\headrulecolor}[1]{\patchcmd{\headrule}{\hrule}{\color{#1}\hrule}{}{}}
\newcommand{\footrulecolor}[1]{\patchcmd{\footrule}{\hrule}{\color{#1}\hrule}{}{}}
\renewcommand{\headrulewidth}{1pt}
\renewcommand{\footrulewidth}{1pt}
\headrulecolor{redisen}
\footrulecolor{redisen}


\pagestyle{fancy}
% pour le haut de page
\lhead{\textcolor{grayisen1}{\longtitle}}
\chead{}
\makeatletter
\rhead{\textcolor{grayisen1}{\shorttitle}}
\makeatother
% pour le pied de page
\cfoot{\textcolor{grayisen1}{Page \thepage/\pageref{LastPage}}}
\lfoot{\textcolor{grayisen1}{\laclasse}}
\rfoot{\textcolor{grayisen1}{\ecole}}

%%%%%%%%%%%%%%%%%%%%%%%%%%%%%%%%%%%%%%%%%%%
\newtheorem{theoreme}{Th\'eor\`eme}[section]
\newtheorem{lemme}{Lemme}[section]
\newtheorem{corollaire}{Corollaire}[section]
\newtheorem{exemple}{Exemple}[section]
\newtheorem{Def}{D\'efinition}[section]
\newtheorem{remarque}{Remarque}[section]
\newtheorem{Exercice}{Exercice}
%%%%%%%%%%%%%%%%%%%%%%%%%%%%%%%%%%%%%%%%%%%
\def\separateur{\FloatBarrier\begin{center}\textcolor{redisen}{\rule{0.7\linewidth}{1pt}}\end{center}\FloatBarrier}

%%%%%%%%%%%%%%%%%%%%%%%%%%%%%%%%%%%%%%%%%%%%
% Definition of blocks:
\tikzset{%
 block/.style    = {draw, thick, rectangle, minimum height = 3em,
  minimum width = 3em},
 cross/.style={path picture={
  \draw[black]
  (path picture bounding box.south east) -- (path picture bounding box.north west) (path picture bounding box.south west) -- (path picture bounding box.north east);
 }},
 sum/.style      = {draw, circle, node distance = 2cm}, % Adder
 input/.style    = {coordinate}, % Input
 output/.style   = {coordinate}, % Output
 comparo/.style      = {draw, circle, cross, node distance = 2cm} % Adder
}

% %%%%%%%%%%%%%%%%%%%%%%%%%%%%%%%%%%%%%%%%%%%%
% % Definition of blocks:
% \tikzset{%
% 	block/.style    = {draw, thick, rectangle, minimum height = 3em,
% 		minimum width = 3em},
% 	cross/.style={path picture={
% 		\draw[black]
% 		(path picture bounding box.south east) -- (path picture bounding box.north west) (path picture bounding box.south west) -- (path picture bounding box.north east);
% 	}},
% 	comparo/.style      = {draw, circle, cross, node distance = 2cm}, % Adder
% 	input/.style    = {coordinate}, % Input
% 	output/.style   = {coordinate} % Output
% }


\graphicspath{{../images/}}
\makeatletter
\def\myauthor{\@author}
\def\mytitle{\@title}
\makeatother
\newcommand\makeMyTitle{%\ladate
 %%%%%%%%%%%%%%%%%%%%%%%%%%%%%%%%%%%%%%%%%%%%
 \begin{figure}[!ht]
  \vspace{-0.5cm}
  \begin{minipage}{6cm}
   \hspace{1cm}
   \includegraphics[width=6cm]{/ISEN-Brest_horizontal.jpg}
   % \vspace{0.5cm}
   \begin{center}\hspace{1.5cm}\textcolor{grayisen1}{Année \ladate}\end{center}
  \end{minipage}%
  \hspace{2cm}
  \begin{minipage}{6cm}\begin{center}\textcolor{grayisen1}{
    \myauthor\\
    \ifx\intervenantTD\undefined
     %% Do this if it is undefined
    \else
     %% Do this if it is defined
     \vspace{0.25cm}
     Intervenant TD : \intervenantTD
    \fi
   }
   \end{center}
  \end{minipage}
 \end{figure}
 \vspace{-1cm}
 \begin{center}\textcolor{redisen}{\rule{\linewidth}{1pt}}\end{center}\FloatBarrier
 \begin{center}\huge{\vspace{-0.5cm}\textcolor{grayisen1}{\linespread{1.5}\mytitle}\vspace{-0.5cm}}\par\end{center}
 \begin{center}\textcolor{redisen}{\rule{\linewidth}{1pt}}\end{center}\FloatBarrier
}
% \makeatother
